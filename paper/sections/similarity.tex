\section{Similarity Metric}\label{sec:similarity}

Our goal is to construct \rtec\ event descriptions that accurately define the composite activities of the domain.
%
For this reason, we compare the event descriptions generated by LLMs with hand-crafted event descriptions constructed by domain experts.
%
To do this, we define a similarity metric for event descriptions.
%
Our similiarity metric extends the metric proposed in~\cite{DBLP:journals/ml/MichelioudakisA19}, which computes the similarity between collections of ground atoms.
%
First, we outline the similarity metric of~\cite{DBLP:journals/ml/MichelioudakisA19}.
%
Subsequently, we extend the metric in order to be suitable for logic programs, which possibly include variables.

\begin{definition}[Distance between Ground Expressions]\label{def:ground_atom_dist}
%
%
\qeddef
%
\end{definition}

\begin{definition}[Distance between Collections of Ground Atoms]\label{def:coll_ground_atoms_dist}

%
\qeddef
%
\end{definition}

We define the distance between two rules in logic programming as follows.

\begin{definition}[Rule Distance]\label{def:rule_dist}

%
\qeddef
%
\end{definition}

Based on Definition \ref{def:rule_dist}, we handle atoms with variables by...


We define the distance between two event descriptions as follows.

\begin{definition}[Event Description Distance]\label{def:ed_dist}

%
\qeddef
%
\end{definition}

The similarity of two event descriptions with distance $d$ is $1\minus d$.
